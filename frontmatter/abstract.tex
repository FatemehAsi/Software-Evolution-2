\chapter*{چکیده}
\addcontentsline{toc}{chapter}{چکیده}

گزارش حاضر با هدف تحلیل معماری دفاعی نرم‌افزار، به بررسی روش‌های متقابل در برابر تحلیل کدهای اجرایی می‌پردازد. با توجه به اهمیت حیاتی جلوگیری از مهندسی معکوس و سوءاستفاده از مالکیت فکری و آسیب‌پذیری‌های امنیتی، توسعه‌دهندگان به سمت استفاده از تکنیک‌های ضد دیس‌اسمبلی (AD) سوق داده شده‌اند. در این راستا، تکنیک‌های اصلی AD شامل ابهام‌سازی کد در سطوح واژگانی، داده‌ای و جریان کنترل، بهره‌گیری از دستورات و ساختارهای غیرمعمول و وابسته به حالت پردازنده برای شکستن تحلیل استاتیک، روش‌های پنهان‌سازی مبتنی بر تجزیه و تحلیل پویا نظیر رمزگشایی زمان اجرا و تکنیک‌های ضد دیباگ، و همچنین تکنیک‌های رمزگذاری مانند کدهای پلی‌مورفیک و متامورفیک مورد بررسی قرار می‌گیرند.

این گزارش به فلسفه و تکنیک‌های ضد ضد دیس‌اسمبلی (AADA) می‌پردازد که در پاسخ به پیچیدگی‌های AD توسط تحلیل‌گران امنیتی ایجاد شده‌اند. این تکنیک‌ها شامل تحلیل دینامیک و رصد کد برای کشف مسیر اجرای واقعی، استفاده از دیس‌اسمبلرهای پیشرفته با قابلیت‌های تحلیل هوشمند و اسکریپت‌نویسی، بهره‌گیری از اجرای نمادین و شبیه‌سازی داینامیک برای کاوش در مسیرهای اجرایی مبهم، و وصله‌زدن و اصلاح کد برای بازسازی نمودار جریان کنترل است.

نتایج نشان می‌دهند که مقابله با مهندسی معکوس یک «مسابقه تسلیحاتی» دائمی میان روش‌های دفاعی AD و روش‌های تحلیلی AADA است. موفقیت در حفظ امنیت و تکامل نرم‌افزار، مستلزم درک عمیق از هر دو جبهه و استفاده هوشمندانه و ترکیبی از ابزارهای تحلیل استاتیک و دینامیک برای خنثی‌سازی مؤثر لایه‌های ابهام‌سازی است.
\newpage

